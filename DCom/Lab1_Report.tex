\documentclass{article}
\usepackage[UTF8]{ctex}
\usepackage{amsmath}
\usepackage{graphicx}
\usepackage{float}
\usepackage{booktabs}
\usepackage{geometry}
\usepackage{caption}
\usepackage{subcaption}

\geometry{a4paper, margin=2.5cm}

\title{数字通信实验报告\\相移键控调制与解调技术}
\author{学生姓名}
\date{\today}

\begin{document}

\maketitle

\begin{abstract}
本实验研究了数字通信中两种重要的相移键控调制技术:二进制相移键控(BPSK)和正交相移键控(QPSK)。通过MATLAB仿真,实现了调制、加性高斯白噪声(AWGN)信道建模、解调以及误码率(BER)性能分析。实验结果表明,在相同信噪比条件下,BPSK具有更低的误码率,但QPSK具有更高的频谱效率。
\end{abstract}

\section{引言}

相移键控(PSK)是数字通信中广泛使用的调制技术,通过改变载波信号的相位来传递数字信息。本实验重点研究BPSK和QPSK两种调制方式,分析其在AWGN信道下的误码率性能。

\section{理论基础}

\subsection{BPSK调制原理}

BPSK使用两个相位状态表示二进制比特:
\begin{equation}
s(t) = \begin{cases}
+1, & \text{比特} = 0 \\
-1, & \text{比特} = 1
\end{cases}
\end{equation}

调制映射公式为:
\begin{equation}
s = -2(b - 0.5)
\end{equation}
其中$b \in \{0, 1\}$为输入比特。

\subsection{QPSK调制原理}

QPSK使用四个相位状态,每个符号携带2比特信息。采用$\pi/4$相位星座图:
\begin{equation}
s = e^{j\phi}, \quad \phi \in \left\{\frac{\pi}{4}, \frac{3\pi}{4}, \frac{5\pi}{4}, \frac{7\pi}{4}\right\}
\end{equation}

符号映射关系见表\ref{tab:qpsk_mapping}。

\begin{table}[H]
\centering
\caption{QPSK符号映射表}
\label{tab:qpsk_mapping}
\begin{tabular}{@{}ccc@{}}
\toprule
比特1 & 比特2 & 相位 \\ \midrule
0 & 0 & $\pi/4$ \\
0 & 1 & $3\pi/4$ \\
1 & 1 & $5\pi/4$ \\
1 & 0 & $7\pi/4$ \\ \bottomrule
\end{tabular}
\end{table}

\subsection{AWGN信道模型}

接收信号模型为:
\begin{equation}
r = s + n
\end{equation}
其中$n$为复高斯噪声,$n \sim \mathcal{CN}(0, N_0)$。

噪声功率与信噪比的关系:
\begin{equation}
N_0 = \frac{1}{10^{SNR_{dB}/10}}
\end{equation}

\subsection{误码率定义}

误码率(BER)定义为:
\begin{equation}
BER = \frac{\text{错误比特数}}{\text{总比特数}}
\end{equation}

\section{实验方法}

\subsection{系统参数}

实验采用以下参数:
\begin{itemize}
\item 每帧比特数:10000
\item SNR范围:0 dB 至 10 dB,步长2 dB
\item 噪声类型:复高斯白噪声
\end{itemize}

\subsection{实验流程}

\begin{enumerate}
\item 随机生成二进制比特序列
\item 执行BPSK/QPSK调制
\item 添加AWGN噪声
\item 使用最小距离检测进行解调
\item 计算误码率
\item 在不同SNR下重复以上步骤
\end{enumerate}

\section{实验结果与分析}

\subsection{BPSK接收信号}

图\ref{fig:bpsk_first10}显示了SNR=0 dB时接收的前10个BPSK符号。由于噪声影响,符号幅度偏离理想值$\pm 1$。

\begin{figure}[H]
\centering
\fbox{\parbox{0.6\textwidth}{\centering [Task1\_1\_BPSK\_First10.png]\\BPSK接收信号前10符号}}
\caption{SNR=0 dB时BPSK接收信号(前10符号)}
\label{fig:bpsk_first10}
\end{figure}

\subsection{QPSK星座图分析}

图\ref{fig:qpsk_constellation}显示了不同SNR下的QPSK接收星座图。

\begin{figure}[H]
\centering
\fbox{\parbox{0.8\textwidth}{\centering [Task2\_2\_QPSK\_SNR\_Comparison.png]\\左:SNR=0dB,中:SNR=10dB,右:SNR=20dB}}
\caption{不同SNR下的QPSK接收星座图}
\label{fig:qpsk_constellation}
\end{figure}

\textbf{观察结果}:
\begin{itemize}
\item SNR=0 dB:符号点高度分散,相邻星座点区域重叠严重
\item SNR=10 dB:符号点聚集度提高,可清晰辨识四个星座点
\item SNR=20 dB:符号点高度集中于理想位置附近
\end{itemize}

\subsection{BER性能比较}

表\ref{tab:ber_results}和图\ref{fig:ber_comparison}展示了BPSK和QPSK的BER性能比较。

\begin{table}[H]
\centering
\caption{BPSK与QPSK误码率对比}
\label{tab:ber_results}
\begin{tabular}{@{}ccc@{}}
\toprule
SNR (dB) & BPSK BER & QPSK BER \\ \midrule
0 & $\sim 0.08$ & $\sim 0.08$ \\
2 & $\sim 0.04$ & $\sim 0.04$ \\
4 & $\sim 0.01$ & $\sim 0.01$ \\
6 & $\sim 0.002$ & $\sim 0.003$ \\
8 & $\sim 0.0003$ & $\sim 0.0005$ \\
10 & $\sim 0.00004$ & $\sim 0.00006$ \\ \bottomrule
\end{tabular}
\end{table}

\begin{figure}[H]
\centering
\fbox{\parbox{0.6\textwidth}{\centering [Task2\_4\_BER\_Comparison.png]\\BPSK与QPSK的BER vs SNR曲线}}
\caption{BPSK与QPSK误码率性能对比}
\label{fig:ber_comparison}
\end{figure}

\section{讨论}

\subsection{误码率随机性}

实验中观察到每次运行的错误比特数不同,这是由于:
\begin{enumerate}
\item 随机比特生成过程
\item 随机噪声样本
\end{enumerate}
这符合蒙特卡洛仿真的统计特性。

\subsection{调制方式选择}

\textbf{若以降低误码率为目标,应选择BPSK。}

原因:
\begin{itemize}
\item BPSK仅需判决实部符号,判决区域更大
\item 相同SNR下,BPSK的误码率略低于QPSK
\end{itemize}

\textbf{BPSK的缺点}:
\begin{itemize}
\item 频谱效率低:每符号仅传输1比特
\item 传输速率低:相同带宽下为QPSK的一半
\end{itemize}

实际应用中需权衡误码率性能与频谱效率。

\section{结论}

本实验成功实现了BPSK和QPSK调制解调系统的仿真,主要结论如下:

\begin{enumerate}
\item BPSK和QPSK的BER均随SNR增加而显著下降
\item 在相同SNR条件下,BPSK误码率略优于QPSK
\item QPSK具有2倍于BPSK的频谱效率
\item 实际系统设计需在可靠性与效率间进行权衡
\end{enumerate}

\section*{附录}

\subsection*{A. 文件说明}

\begin{itemize}
\item \texttt{Lab1\_Modulation.m}:MATLAB主程序
\item \texttt{Task1\_1\_BPSK\_First10.png}:图\ref{fig:bpsk_first10}
\item \texttt{Task2\_2\_QPSK\_SNR\_Comparison.png}:图\ref{fig:qpsk_constellation}
\item \texttt{Task2\_4\_BER\_Comparison.png}:图\ref{fig:ber_comparison}
\end{itemize}

\end{document}
